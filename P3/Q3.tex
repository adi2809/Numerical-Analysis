\documentclass[12pt,letterpaper]{article}
\usepackage{fullpage}
\usepackage[top=2cm, bottom=4.5cm, left=2.5cm, right=2.5cm]{geometry}
\usepackage{amsmath,amsthm,amsfonts,amssymb,amscd}
\usepackage{lastpage}
\usepackage{enumerate}
\usepackage{fancyhdr}
\usepackage{mathrsfs}
\usepackage{xcolor}
\usepackage{graphicx}
\usepackage{listings}
\usepackage{hyperref}

\hypersetup{%
  colorlinks=true,
  linkcolor=blue,
  linkbordercolor={0 0 1}
}
 
\renewcommand\lstlistingname{Algorithm}
\renewcommand\lstlistlistingname{Algorithms}
\def\lstlistingautorefname{Alg.}

\lstdefinestyle{Python}{
    language        = Python,
    frame           = lines, 
    basicstyle      = \footnotesize,
    keywordstyle    = \color{blue},
    stringstyle     = \color{green},
    commentstyle    = \color{red}\ttfamily
}

\setlength{\parindent}{0.0in}
\setlength{\parskip}{0.05in}

% Edit these as appropriate
\newcommand\course{APL 452}
\newcommand\hwnumber{1}                  % <-- homework number
\newcommand\NetIDa{2019ME10770}           % <-- NetID of person #1
\newcommand\NetIDb{Aditya Shankar Garg}           % <-- NetID of person #2 (Comment this line out for problem sets)

\pagestyle{fancyplain}
\headheight 35pt
\lhead{\NetIDa}
\lhead{\NetIDa\\\NetIDb}                 % <-- Comment this line out for problem sets (make sure you are person #1)
\chead{\textbf{\Large Homework \hwnumber}}
\rhead{\course \\ \today}
\lfoot{}
\cfoot{}
\rfoot{\small\thepage}
\headsep 1.5em

\begin{document}

\section*{Problem 3}


\begin{enumerate}
  \item
  \textbf{derive the expression for $s=4$ backward differentiation method}
  \newline
  \newline
   to derive the scheme we need to find $\beta$ which is given by the following expression : 
   \[\beta = \frac{1}{\sum \frac{1}{m}} = \frac{1}{1+\frac{1}{2}+\frac{1}{3}+\frac{1}{4}} = \frac{12}{25}\]
  now we need to write the polynomial that will denote the left hand side of the scheme (also known as the characteristic polynomial).
  \[\rho(w) = \frac{12}{25} \left( w^3(w-1) + \frac{1}{2} w^2 (w-1)^2 + \frac{1}{3} w (w-1)^3 + \frac{1}{4} (w-1)^4\right)    \]
  \[ = w^4 - \frac{48}{25} w^3 + \frac{36}{25}w^2+ \frac{16}{25} w +\frac{3}{25}\]
  hence the left hand side of the scheme can be written as the following explicit formulae: 
  \[  y_{n+4} - \frac{48}{25} y_{n+3} + \frac{36}{25}y_{n+2}+ \frac{16}{25} y_{n+1} +\frac{3}{25} y_{n}\]
  also the right hand side of the scheme can be written as the following explicit formulae: 
  \[\frac{12}{25} h f(t_{n+4}, y_{n+4})\]
  
  the total scheme is given by the following formulae : 
   \[  y_{n+4} - \frac{48}{25} y_{n+3} + \frac{36}{25}y_{n+2}+ \frac{16}{25} y_{n+1} +\frac{3}{25} y_{n} = \frac{12}{25} h f(t_{n+4}, y_{n+4})\]
\end{enumerate}




\end{document}
